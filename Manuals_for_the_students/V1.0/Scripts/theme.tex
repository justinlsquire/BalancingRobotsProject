% ~~~~~~~~~~~~~~~~~~~~~~~~~~~~~~~~~~~~~~~~~~~~~~~~~~~~~~~~~~~~~~~~~~~~~~~~~~~~~~~~~~~~~ %
%																						%
\usetheme			[]							{default}								%
\usefonttheme		[]							{default}								%
\usecolortheme		[]							{default}								%
\useinnertheme		[shadow]					{rounded}								%
\useoutertheme		[]							{default}								%
%																						%
\setbeamercolor		{normal text}				{bg=\SoftSecondary,	fg=\StrongPrimary}	%
\setbeamercolor		{structure}					{bg=\SoftSecondary,	fg=\StrongPrimary}	%
%																						%
\setbeamercolor		{frametitle}				{bg=, fg=\StrongPrimary}				%
\setbeamercolor		{block title}				{bg=\SoftSecondary,	fg=\StrongPrimary}	%
\setbeamercolor		{block title alerted}		{bg=\SoftSecondary,	fg=\StrongPrimary}	%
\setbeamercolor		{block title example}		{bg=\SoftSecondary,	fg=\StrongPrimary}	%
\setbeamercolor		{block body}				{bg=\Background,	fg=\Foreground}		%
\setbeamercolor		{block body alerted}		{bg=\Background,	fg=\Foreground}		%
\setbeamercolor		{block body example}		{bg=\Background,	fg=\Foreground}		%
%																						%
\setbeamercolor		{alerted text}				{bg=\Background,	fg=\StrongPrimary}	%
\setbeamercolor		{section in toc}			{bg=\Background,	fg=\Foreground}		%
\setbeamercolor		{math text}					{bg=\Background,	fg=\Foreground}		%
\setbeamercolor		{math text inlined}			{bg=\Background,	fg=\Foreground}		%
\setbeamercolor		{math text displayed}		{bg=\Background,	fg=\Foreground}		%
\setbeamercolor		{normal text}				{bg=\Background,	fg=\Foreground}		%
\setbeamercolor		{normal text in math text}	{bg=\Background,	fg=\Foreground}		%
%																						%
\setbeamercolor		{item}						{use={structure,normal text}}			%
\setbeamercolor		{background canvas}			{bg=\Background}						%
\setbeamertemplate	{navigation symbols}		{}										%
\setbeamercovered	{transparent=0}														%
\setbeamersize		{description width of={a}}											%
%																						%
% ~~~~~~~~~~~~~~~~~~~~~~~~~~~~~~~~~~~~~~~~~~~~~~~~~~~~~~~~~~~~~~~~~~~~~~~~~~~~~~~~~~~~~ %

\everymath={\displaystyle}

% \useinnertheme	[]{default}
% \useinnertheme	[]{circles}
% \useinnertheme	[]{rectangles}
% \useinnertheme	[]{rounded}
% \useinnertheme	[shadow]{rounded}
% \useinnertheme	[]{inmargin}


% -------------------------------------------------------------------------
% Logo settings
%\logo{\includegraphics[height = 1cm]{logo}}


% -------------------------------------------------------------------------
% if you DO want the navigation symbols you have to comment the following line
\setbeamertemplate{navigation symbols}{}


% -------------------------------------------------------------------------
% slides number
\setbeamertemplate{footline}{\hfill \insertframenumber \vspace{0.1cm} \hspace{0.1cm}} 


% -------------------------------------------------------------------------
% settings of the text to be unshown. Options:
% -> invisible	[text is invisible until it must appear]
% -> trasparent	[text is opaque (in %) until it must appear]
% -> dynamic	[text appears dynamically: initially invisible, then opaque and then appears fully]
\setbeamercovered{transparent=0}


% -------------------------------------------------------------------------
% if the frame is not fully occupied by text put the white space on the bottom
\raggedbottom


\providecommand\thispdfpagelabel[1]{}			% TEMPORARY


% -------------------------------------------------------------------------
% linespread definition
\linespread{1.1}


% -------------------------------------------------------------------------
% in order to remove some useless warnings
\let\Tiny=\tiny


% -------------------------------------------------------------------------
% to have a fancier notes page
\makeatletter
\defbeamertemplate{note page}{lookahead}
{
	\vskip0.5cm
	\begin{tikzpicture}
		\node (note) [text width = 1.3\textwidth, minimum width = 1.3\textwidth, minimum height = 0.85\textheight, draw, rounded corners, align = justify] {\begin{varwidth}{\linewidth} \insertnote \end{varwidth}};
		\node [above = 0cm of note] {\emph{notes}};
	\end{tikzpicture}
}
\makeatother
\setbeamertemplate{note page}[lookahead]


% -------------------------------------------------------------------------
% measurement units:
%
% in - inches
% mm - millimeters
% cm - centimeters
% pt - points (about 1/72 inch)
% em - approximately the width of an "M" in the current font
% ex - approximately the height of an "x" in the current font 
%
% usage:
%\setlength{\thing_to_be_modified}{my_offset}
%
% modifiable things:
%
%--- Page Layout
%\columnsep:		gap between columns
%\topmargin:		gap above header
%\topskip:			between header and text
%\textheight:		height of main text
%\textwidth:		width of text
%\linewidth:		width of a line in the local environment
%\oddsidemargin:	odd page left margin
%\evensidemargin:	even page left margin
%\baselineskip:		normal vertical distance between lines in a paragraph
%\baselinestretch:	multiplies \baselineskip
%\voffset:
%
%--- Paragraphs
%\parindent:		indentation of paragraphs
%\parskip:			gap between paragraphs
%
%--- Floats (tables and figures)
%\floatsep:			space left between floats.
%\textfloatsep:		space between last top float or first bottom float and the text.
%\intextsep:		space left on top and bottom of an in-text float.
%\dbltextfloatsep:	is \textfloatsep for 2 column output.
%\dblfloatsep:		is \floatsep for 2 column output.
%\abovecaptionskip:	space above caption
%\belowcaptionskip:	space below caption
%\unitlength:		units of lenght in Picture Environment 
%
%--- Maths
%\abovedisplayskip:	space before maths
%\belowdisplayskip:	space after maths
%\arraycolsep:		gap between columns of an array
%
%--- Lists
%\topsep:			space between first item and preceding paragraph.
%\partopsep:		extra space added to \topsep when environment starts a new paragraph.
%\itemsep:			space between successive items.


% BACKGROUND:
% DECOMMENT THE PREFERRED OPTION


% -------------------------------------------------------------------
% image
%
% \setbeamertemplate{background}
% {
% 	\centering
% 	{
% 		\includegraphics[width=\paperwidth,height=\paperheight]{background_file.jpg}
% 	}
% }



% -------------------------------------------------------------------
% horizontal shading
%
% \pgfdeclarehorizontalshading
% {horizontal}					% name of the shading
% {2cm}							% shading height
% {	rgb(0cm)=(1.0, 1.0, 1.0);	% initial color
% 	rgb(1cm)=(1.0, 1.0, 0.8)}	% final color
%
% \AddToShipoutPicture
% {
% 	\begin{tikzpicture}[remember picture,overlay,shading=horizontal]
% 		\node (aa)	[xshift=-\textwidth,yshift=-\textheight]	at	(current page.south west)	{};
% 		\node (bb)	[xshift=+\textwidth,yshift=+\textheight]	at	(current page.north east)	{};
% 		\shade[shading angle=-90]	(aa)	rectangle	(bb);
% 	\end{tikzpicture}
% }



% -------------------------------------------------------------------
% radial shading
%
% \pgfdeclareradialshading
% {radial}						% nome
% {\pgfpoint{1.0cm}{0.7cm}}		% posizione del centro di illuminazione (0,0 � in mezzo alla sfera)
% {	rgb(0cm)=(0.9, 0.0, 0.0);	% colore iniziale
% 	rgb(2cm)=(0.5, 0.0, 0.0)}	% colore finale
%
% \AddToShipoutPicture
% {
% 	\begin{tikzpicture}[remember picture,overlay,shading=radial]
% 		\node (aa)	[xshift=-\textwidth,yshift=-\textheight]	at	(current page.south west)	{};
% 		\node (bb)	[xshift=+\textwidth,yshift=+\textheight]	at	(current page.north east)	{};
% 		\shade (aa)	rectangle	(bb);
% 	\end{tikzpicture}
% }



% -------------------------------------------------------------------
% text
%
% \AddToShipoutPicture
% {
% 	\begin{tikzpicture}[remember picture,overlay]
% 		\node [rotate=-60,scale=10,text opacity=0.1]
% 		at (current page.center)
% 		{For peer review only};
% 	\end{tikzpicture}
% }




