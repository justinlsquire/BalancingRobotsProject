
\newcommand{\DefinedAs}			[0]	{\mathrel{\mathop:}=}
\newcommand{\IDefinedAs}		[0]	{=\mathrel{\mathop:}}


\newcommand{\ExponentialOf}		[1]	{\mathrm{exp} \left( #1 \right)}
\newcommand{\LogarithmOf}		[1]	{\mathrm{log} \left( #1 \right)}
\newcommand{\ConvexHullOf}		[1]	{\mathrm{c.h.} \left( #1 \right)}


\newcommand{\MaximumOfOne}		[1]	{\mathrm{max} \left\lbrace #1 \right\rbrace}
\newcommand{\MaximumOfTwo}		[2]	{\mathrm{max} \left\lbrace #1, #2 \right\rbrace)}
\newcommand{\MaximumOfThree}	[3]	{\mathrm{max} \left\lbrace #1, #2, #3 \right\rbrace)}
\newcommand{\MinimumOfOne}		[1]	{\mathrm{min} \left\lbrace #1 \right\rbrace}
\newcommand{\MinimumOfTwo}		[2]	{\mathrm{min} \left\lbrace #1, #2 \right\rbrace)}
\newcommand{\MinimumOfThree}	[3]	{\mathrm{min} \left\lbrace #1, #2, #3 \right\rbrace)}


\newcommand{\CostFunction}				[0]	{Q}
\newcommand{\CostFunctionOf}			[1]	{\CostFunction \left( #1 \right)}
\newcommand{\CostFunctionOfSensor}		[1]	{\CostFunction_{#1}}
\newcommand{\CostFunctionOfSensorOf}	[2]	{\CostFunctionOfSensor{#1} \left( #2 \right)}


\newcommand{\KroneckerDeltaOf}			[2]	{\delta_{#1 #2}}


\newcommand{\FloorOf}			[1]	{\lfloor #1 \rfloor}
\newcommand{\CeilOf}			[1]	{\lceil #1 \rceil}


\newcommand{\GammaFunctionOf}	[1]	{\Gamma \left( #1 \right)}

\newcommand{\IdentityMatrix}		[1]	{I_{#1}}
\newcommand{\OnesVector}			[1]	{\mathds{1}_{#1}}

\newcommand{\TraceOf}				[1]	{\text{tr} \left( #1 \right)}
\newcommand{\DeterminantOf}			[1]	{\text{det} \left( #1 \right)}
\newcommand{\SetOfEigenvaluesOf}	[1]	{\text{eig} \left( #1 \right)}

\newcommand{\Eigenvalue}			[1]	{\lambda_{#1}}
\newcommand{\Eigenvector}			[1]	{v_{#1}}
\newcommand{\MaximalEigenvalue}		[0]	{\lambda_{\text{max}}}
\newcommand{\MinimalEigenvalue}		[0]	{\lambda_{\text{min}}}

\newcommand{\SpectralRadius}		[0]	{\mu}
\newcommand{\SpectralRadiusOf}		[1]	{\SpectralRadius \left( #1 \right)}


\newcommand{\GaussianDistribution}					[2]	{\mathcal{N} \left( #1, #2 \right)}
\newcommand{\GammaDistribution}						[2]	{\text{Gamma} \left( #1, #2 \right)}
\newcommand{\ChiSquareDistribution}					[0]	{\chi^{2}}
\newcommand{\ChiSquareDistributionOfIndex}			[1]	{\ChiSquareDistribution \left( #1 \right)}
\newcommand{\InverseChiSquareDistribution}			[0]	{\text{Inv-}\chi^{2}}
\newcommand{\InverseChiSquareDistributionOfIndex}	[1]	{\InverseChiSquareDistribution \left( #1 \right)}
\newcommand{\UniformDistribution}					[2]	{\mathcal{U} \left[ #1, #2 \right]}
\newcommand{\ExponentialDistribution}				[1]	{\text{Exp} \left( #1 \right)}


\newcommand{\Reals}						[0]	{\mathbb{R}}
\newcommand{\PositiveReals}				[0]	{\mathbb{R}_{+}}
\newcommand{\Naturals}					[0]	{\mathbb{N}}
\newcommand{\PositiveNaturals}			[0]	{\mathbb{N}_{+}}

\newcommand{\SetOfSquareSummableInfiniteVectors}			[0]	{\ell}
\newcommand{\SetOfWeightedSquareSummableInfiniteVectors}	[0]	{\textsl{l}_{\RKHS}}
\newcommand{\SetOfSquareIntegrableFunctions}				[0]	{\textsl{L}^{2}}
\newcommand{\SetOfSquareIntegrableFunctionsIn}				[1]	{\textsl{L}^{2} \left( #1 \right)}

\input{./Scripts/newcommands__statistical_operators}
\newcommand{\Fingers}
{
	\begin{itemize}
		\item yes \hfill (1 finger)
		\item yes, maybe \hfill (2 fingers)
		\item I don't know \hfill (3 fingers)
		\item no, maybe not \hfill (4 fingers)
		\item no \hfill (5 fingers)
	\end{itemize}
}

\newcommand	{\SuchThat}				{s.t.\ }

\newcommand	{\Section}				[0]	{Sec.}
\newcommand	{\Sections}				[0]	{Secc.}
\newcommand	{\Equation}				[0]	{Equ.}
\newcommand	{\Equations}			[0]	{Equu.}
\newcommand	{\Figure}				[0]	{Fig.}
\newcommand	{\Figures}				[0]	{Figg.}
\newcommand	{\Table}				[0]	{Tab.}
\newcommand	{\Tables}				[0]	{Tabb.}
\newcommand	{\Algorithm}			[0]	{Alg.}
\newcommand	{\Algorithms}			[0]	{Algg.}
\newcommand	{\Proposition}			[0]	{Prop.}
\newcommand	{\Propositions}			[0]	{Propp.}
\newcommand	{\Hypothesis}			[0]	{Hyp.}
\newcommand	{\Hypotheses}			[0]	{Hypp.}

\setbeamertemplate{theorems}[numbered]
\newtheorem{question}{Question}

\newcounter{QuestionsCounter}
\stepcounter{QuestionsCounter}

\newcommand	{\QuestionID}		[1]	{\JustSecondary{(ID: #1)} \label{question:#1} \stepcounter{QuestionsCounter}}
\newcommand	{\QuestionKCs}		[1] {}
\newcommand	{\QuestionKCsTaxonomies}	[1] {}
\newcommand	{\QuestionNotes}	[1] {}
\newcommand	{\QuestionBody}		[1] {#1}
\newcommand	{\QuestionImage}	[2] {\begin{center} \includegraphics[#1]{#2} \end{center}}
\newcommand	{\QuestionAnswers}	[1] {\begin{enumerate} #1 \end{enumerate}}
\newcommand	{\QuestionSolution}	[1] {\ifshowsolutions \note<1->{#1} \fi}
\newcommand	{\QuestionAuthor}	[1] {}
\newcommand	{\QuestionVersion}	[1] {}







\newcommand{\answer}		[0]
{
	\item[\addtocounter{enumi}{1} \tikz{ \node [anchor = base, baseline, minimum width = 0.6cm, minimum height = 0.6cm, draw] {\arabic{enumi}};}]
}

\newcommand{\correctanswer}	[0]
{
	\ifshowsolutions
		\item[\addtocounter{enumi}{1} \tikz{ \node [anchor = base, baseline, minimum width = 0.6cm, minimum height = 0.6cm, draw, fill = red!30!white] {\arabic{enumi}};}]
	\else
		\answer
	\fi
}

% \newcommand{\hideableanswer}[1]{\item \ifshowsolutions #1 \fi}





\pgfdeclarelayer{ultrabackground}
\pgfdeclarelayer{background}
\pgfdeclarelayer{foreground}
\pgfsetlayers{ultrabackground,background,main,foreground}
% \begin{pgfonlayer}{background} 
% \end{pgfonlayer}


\newcommand{\JustPrimary}		[1]{\textcolor{\StrongPrimary}{#1}}
\newcommand{\ItPrimary}			[1]{\textcolor{\StrongPrimary}{\textit{#1}}}
\newcommand{\BoldPrimary}		[1]{\textcolor{\StrongPrimary}{\textbf{#1}}}
\newcommand{\BoldItPrimary}		[1]{\textcolor{\StrongPrimary}{\textit{ \textbf{#1} }}}
\newcommand{\ItBoldPrimary}		[1]{\BoldItPrimary{#1}}
\newcommand{\MathBoxPrimary}	[1]{\colorbox{\SoftPrimary}{#1}}
%
\newcommand{\JustSecondary}		[1]{\textcolor{\StrongSecondary}{#1}}
\newcommand{\ItSecondary}		[1]{\textcolor{\StrongSecondary}{\textit{#1}}}
\newcommand{\BoldSecondary}		[1]{\textcolor{\StrongSecondary}{\textbf{#1}}}
\newcommand{\BoldItSecondary}	[1]{\textcolor{\StrongSecondary}{\textit{ \textbf{#1} }}}
\newcommand{\ItBoldSecondary}	[1]{\BoldItSecondary{#1}}
\newcommand{\MathBoxSecondary}	[1]{\colorbox{\SoftSecondary}{#1}}
%
\newcommand{\Code}				[1]{\texttt{\textcolor{\StrongSecondary}{#1}}}


\newcommand{\NewSection}		[1]
{
	\subsection{#1}
	\label{sec: #1}
	\setbeamercolor{background canvas}{bg=\SoftPrimary}
	\begin{frame}
		\begin{center}
			\Large #1
		\end{center}
		\note<1-1>{\begin{itemize}
			\item 
		\end{itemize}}
	\end{frame}
	\setbeamercolor{background canvas}{bg=white}
}

\newcommand{\QuestionMark}		[0]
{
	\setbeamercolor{background canvas}{bg=\SoftSecondary}
	\begin{frame}
		\begin{center}
			\Large ?
		\end{center}
		\note<1-1>{\begin{itemize}
			\item 
		\end{itemize}}
	\end{frame}
	\setbeamercolor{background canvas}{bg=white}
}

\newcommand {\PrimaryRectangle} [1]
{
	\begin{center}
	\begin{tikzpicture}
		%
		\node
		[
			shape			= rectangle,		% shape
			rounded corners	= 0.2cm,			% shape
			minimum width	= 0.7cm,			%
			minimum height	= 0.7cm,			%
			line width		= 0cm,				% thickness of the border
			fill			= \SoftPrimary,		%
			draw			= \StrongPrimary,	% draw the border with this color
			line width		= 0.1cm,			% thickness
			text width		= 0.8\textwidth,	% max. width of the text
			align			= center,			% text alignment
			inner xsep		= 0.2cm,			%
			inner ysep		= 0.2cm,			%
		]
		{#1};
		%
	\end{tikzpicture}
	\end{center}
}

\newcommand {\SecondaryRectangle} [1]
{
	\begin{center}
	\begin{tikzpicture}
		%
		\node
		[
			shape			= rectangle,		% shape
			rounded corners	= 0.2cm,			% shape
			minimum width	= 0.7cm,			%
			minimum height	= 0.7cm,			%
			line width		= 0cm,				% thickness of the border
			fill			= \SoftSecondary,	%
			draw			= \StrongSecondary,	% draw the border with this color
			line width		= 0.1cm,			% thickness
			text width		= 0.9\textwidth,	% max. width of the text
			align			= center,			% text alignment
			inner xsep		= 0.2cm,			%
			inner ysep		= 0.2cm,			%
		]
		{#1};
		%
	\end{tikzpicture}
	\end{center}
}

\newcommand {\PrimaryRectangleWithCaption} [3]
{
	\begin{center}
	\begin{tikzpicture}
		%
		\node (a)
		[
			shape			= rectangle,		% shape
			rounded corners	= 0.2cm,			% shape
			minimum width	= 0.7cm,			%
			minimum height	= 0.7cm,			%
			fill			= \SoftPrimary,		%
			draw			= \StrongPrimary,	% draw the border with this color
			line width		= 0.1cm,			% thickness
			text width		= 0.8\textwidth,	% max. width of the text
			align			= center,			% text alignment
			inner xsep		= 0.3cm,			%
			inner ysep		= 0.3cm,			%
		]
		{#2};
		%
		\node
		[
			shape			= rectangle,		% shape
			rounded corners	= 0.2cm,			% shape
			anchor			= mid,
			fill			= \SoftPrimary,		%
			draw			= \StrongPrimary,	% draw the border with this color
			text			= \StrongPrimary,	%
			align			= center,			% text alignment
			line width		= 0.1cm,			% thickness
			inner xsep		= 0.2cm,			%
			inner ysep		= 0.2cm,			%
		]
		at (a.#3)
		{#1};
		%
	\end{tikzpicture}
	\end{center}
}

\newcommand {\SecondaryRectangleWithCaption} [3]
{
	\begin{center}
	\begin{tikzpicture}
		%
		\node (a)
		[
			shape			= rectangle,		% shape
			rounded corners	= 0.2cm,			% shape
			minimum width	= 0.7cm,			%
			minimum height	= 0.7cm,			%
			fill			= \SoftSecondary,	%
			draw			= \StrongSecondary,	% draw the border with this color
			line width		= 0.1cm,			% thickness
			text width		= 0.8\textwidth,	% max. width of the text
			align			= center,			% text alignment
			inner xsep		= 0.3cm,			%
			inner ysep		= 0.3cm,			%
		]
		{#2};
		%
		\node
		[
			shape			= rectangle,		% shape
			rounded corners	= 0.2cm,			% shape
			anchor			= mid,
			fill			= \SoftSecondary,	%
			draw			= \StrongSecondary,	% draw the border with this color
			text			= \StrongSecondary,	%
			align			= center,			% text alignment
			line width		= 0.1cm,			% thickness
			inner xsep		= 0.2cm,			%
			inner ysep		= 0.2cm,			%
		]
		at (a.#3)
		{#1};
		%
	\end{tikzpicture}
	\end{center}
}

\newcommand{\InsertImage}[3] % path / height / width
{
	\begin{tikzpicture}[remember picture, overlay]
	\node
	[
		shape			= rectangle,		% shape
		minimum height	= #2cm,				% | minimum size of the node
		minimum width	= #3cm,				% |
 		path picture	=
		{\node at (path picture bounding box.center)
		{\includegraphics[height = #2cm, width = #3cm]
		{#1}};}
	]{};
	\end{tikzpicture}
}

\newcommand{\InsertImageAt}[5] % path / height / width / xshift / yshift
{
	\begin{tikzpicture}[remember picture, overlay]
	\node
	[
		shape			= rectangle,		% shape
		minimum height	= #2cm,				% | minimum size of the node
		minimum width	= #3cm,				% |
		xshift 			= #4cm,
		yshift			= #5cm,
 		path picture	=
		{\node at (path picture bounding box.center)
		{\includegraphics[height = #2cm, width = #3cm]
		{#1}};}
	]
	at (current page.center)
	{};
	\end{tikzpicture}
}

\newcommand{\InsertTextAt}[3] % text / xshift / yshift
{
	\begin{tikzpicture}[remember picture, overlay]
	\node
	[
		shape			= rectangle,		% shape
		xshift 			= #2cm,
		yshift			= #3cm,
	]
	at (current page.center)
	{#1};
	\end{tikzpicture}
}

\newcommand{\Formula}[2]
{
	\begin{frame}[t]
	\ifexternalize
		\tikzsetnextfilename{formula-#1}
	\fi
	\begin{tikzpicture}
		\node
		{$
			#2
		$};
	\end{tikzpicture}
	\end{frame}
}

